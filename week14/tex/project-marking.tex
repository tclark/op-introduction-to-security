\documentclass{article}
\usepackage{graphicx}
\usepackage{enumerate}
\usepackage{verbatim}
\usepackage{hyperref}
\usepackage[parfill]{parskip}
\usepackage[margin = 2.5cm]{geometry}

\usepackage[T1]{fontenc}


\begin{document}

\title{ Security Project: Marking Schedule\\ IN618 Security}
\date{}
\maketitle

On this project your assessment will be based on the correctness, thoroughness, and quality of the report document that you submit. The following rubric will be used for marking.

\textbf{Application testing}

\begin{tabular}{|l|l|} 
	\hline
	All inputs and actions tested thoroughly     & 20  marks \\ \hline
	Most tested                                  & 15 \\ \hline
	Some tested, or some gaps in testing		 & 10 \\  \hline
	Testing significantly lacking				 &  5 \\ \hline	
\end{tabular}

\textbf{Code review}

\begin{tabular}{|l|l|} 
	\hline
	All vulnerabilities (or their lack) identified in code     & 10  marks \\ \hline
	Most identified                                            & 7 \\ \hline
	Some identified		                                       & 4 \\  \hline
	Few identified			                    			   & 2 \\ \hline	
\end{tabular}

\textbf{Analysis of user data handling}

\begin{tabular}{|l|l|} 
	\hline
	Complete and accurate analysis                             & 5  marks \\ \hline
	Mainly complete and accurate                               & 3 \\ \hline
	Analysis significantly lacking		                       & 1 \\  \hline	
\end{tabular}

\textbf{Server configuration analysis}

\begin{tabular}{|l|l|} 
	\hline
	Complete and accurate analysis                             & 5  marks \\ \hline
	Mainly complete and accurate                               & 3 \\ \hline
	Analysis significantly lacking		                       & 1 \\  \hline	
\end{tabular}

\textbf{Recommendations for remediation}

\begin{tabular}{|l|l|} 
	\hline
	Comprehensive recommendations provided                     & 5  marks \\ \hline
	Some useful recommendations provided                       & 3 \\ \hline
	Major gaps in recommendations         		               & 1 \\  \hline	
\end{tabular}

\textbf{Document quality} 

It is expected that your report document will be well formatted, clearly and concisely written, and free of errors in spelling, punctuation, and grammar.

\begin{tabular}{|l|l|} 
	\hline
	Document fully meets standards                         & 5  marks \\ \hline
	Document largely meets standards                       & 3 \\ \hline
	Significant problems with document quality         	   & 1 \\  \hline	
\end{tabular}




\end{document}
