\documentclass{article}

\usepackage{hyperref}
\usepackage{verbatim}
\usepackage[parfill]{parskip}
\usepackage[margin = 2.5cm]{geometry}

\usepackage[T1]{fontenc}


\begin{document}

\title{Cross Site Scripting: Introduction\\ IN618 Security}
\date{}
\maketitle

\section*{Introduction}
I discussed this paper with an industry security consultant who told me that 
\emph{cross site scripting} (XSS) was the most important topic we would cover this 
semester. This is because mistakes leading to XSS vulnerabilities are easily
and frequently made, and because they are easily exploited. 

\section*{Examine a vulnerable system}
Go to the web site at \url{http://xss.foo.org.nz} and try out the web 
pages there.  View the HTML source and do anything else you want to 
do to explore the site.

Note that the results you see vary considerably depending on the browser you use, so you may want to try things in a variety of browsers.

1. Do you see any potential vulnerabilities here?  Document your
suspicions below:

\vspace{60mm}

2. Now enter \url{http://xss.foo.org.nz/pg2.php?secure=<script>alert('pwned')</script>}.  What happens?  Why?

\vspace{60mm}

3. Can you think of any other ways to exploit this?  Try some and 
document this below.

\vspace{60mm}



4. Now go to \url{http://xss.foo.org.nz/trouble.html}.  Follow the 
link and see what happens.  Explain what you observe below.

\vspace{90mm}

5. Source code for this example is in the \texttt{week03} directory on GitHub.
Inspect the source and be prepared to discuss it.
\end{document}
