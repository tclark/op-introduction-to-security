% Beamer slide template prepared by Tom Clark <tom.clark@op.ac.nz>
% Otago Polytechnic
% Dec 2012

\documentclass[10pt]{beamer}
\usetheme{Dunedin}
\usepackage{graphicx}
\usepackage{fancyvrb}

\newcommand\codeHighlight[1]{\textcolor[rgb]{1,0,0}{\textbf{#1}}}

\title{Mitigating SQL Injection Vulnerability with Prepared Statements}

\author[IN618]{Security}
\institute[Otago Polytechnic]{
  Otago Polytechnic \\
  Dunedin, New Zealand \\
}
\date{}
\begin{document}

%----------- titlepage ----------------------------------------------%
\begin{frame}[plain]
  \titlepage
\end{frame}



\begin{frame}[fragile]
	\frametitle{Do you want sql injection?}
	\begin{verbatim}
    $id = $_GET['id'];
    $query = "SELECT * FROM users WHERE id=$id";
    $db->query($query);
	\end{verbatim}
	
	Because that's how you get SQL injection!
\end{frame}


\begin{frame}[fragile]
	\frametitle{Step 1}
	\begin{Verbatim}[commandchars=\\\{\}]
	\codeHighlight{$id = $_GET['id'];}
	
	\end{Verbatim}
	
	Take untrusted input,\ldots
\end{frame}

\begin{frame}[fragile]
	\frametitle{Step 2}
	\begin{Verbatim}[commandchars=\\\{\}]
	$id = $_GET['id'];
	\codeHighlight{$query = "SELECT * FROM users WHERE id=$id";}
	
	\end{Verbatim}
	
	\ldots construct an SQL query string with the input,\ldots
\end{frame}

\begin{frame}[fragile]
	\frametitle{Step 3}
	\begin{Verbatim}[commandchars=\\\{\}]
	$id = $_GET['id'];
	$query = "SELECT * FROM users WHERE id=$id";
	\codeHighlight{$db->query($query);}
	
	\end{Verbatim}
	
	\ldots send the query string to the database. Now you're on your way to executing somebody else's code on your server
\end{frame}

\begin{frame}[fragile]
	\frametitle{Mitigating the risk}
	
	First, we should \textbf{never} just accept untrustworthy input.
	
	\begin{verbatim}
	$id = $_GET['id'];
	\end{verbatim}
	
	becomes
	
	\begin{verbatim}
	if(is_numeric($_GET['id'])) {
	    $id =(int) $_GET['id'];
	}
	\end{verbatim}
	
	since id's are meant to be integers in our application.
	
	
\end{frame}

\begin{frame}[fragile]
	\frametitle{Mitigating the risk}
	
	Next, instead of constructing a query string, we create a \emph{prepared statement}.
	
	\begin{verbatim}
	$query = "SELECT * FROM users WHERE id=$id";
	\end{verbatim}
	
	becomes
	
	\begin{verbatim}
$stmt = $db->prepare("SELECT * FROM users WHERE id=?")
	\end{verbatim}
	
	
	
	
\end{frame}

\begin{frame}[fragile]
	\frametitle{Mitigating the risk}
	
	Finally, we bind the variables in our prepared statement and execute it.
	
	\begin{verbatim}
	$db->query($query);
	\end{verbatim}
	
	becomes
	
	\begin{verbatim}
	$stmt->bind_param("i", $id);
	$stmt->execute();
	\end{verbatim}
	
\end{frame}

\begin{frame}
	\frametitle{One more thing \ldots}
	
	We can create database user accounts with varying levels of privileges. Set up and use accounts with the least amount of privileges needed.
	\vspace{5mm}
	
	For example, if your application only needs to read from the database, connect as a user that only has read privileges.
\end{frame}

\begin{frame}
	\frametitle{Today's Lab}
	
	\begin{enumerate}
		\item Download code for this lab from \\
		\url{http://sec-student.foo.org.nz/~tclark/lab7.1.tgz}
		\item The file \texttt{login.php} has already be modified to use
		prepared SQL statements. Use it as an example.
		\item Modify \texttt{reset.php} to use prepared SQL statements.
		\item Additional information is available at \\ 
		\url{http://php.net/manual/en/mysqli.quickstart.prepared-statements.php}
	\end{enumerate}
	
	As always, we use PHP here as an example, but similar techniques apply to other languages.
		
	
	\end{frame}
	

\end{document}

