\documentclass{article}
\usepackage{graphicx}
\usepackage{enumerate}
\usepackage{verbatim}
\usepackage{hyperref}
\usepackage[parfill]{parskip}
\usepackage[margin = 2.5cm]{geometry}

\usepackage[T1]{fontenc}


\begin{document}

\title{ Lab 13.1 Firewalling with iptables\\ IN618 Security}
\date{\today}
\maketitle

\section*{Introduction}
A fundamental task in securing a networked host is the configure and run a firewall.  In this lab we will see how to configure \texttt{iptables} to provide firewalling on  Linux server.

\section*{Procedure}

\begin{enumerate}
	\item Obtain the ip address of a lab virtual machine from the lecturer.  Ssh into it with our standard username/password.
	\item Install apache2 and nmap with the commands \\
	\begin{verbatim}
	sudo apt-get update
	sudo apt-get install apache2 nmap
	\end{verbatim}
	You may need to modify your \texttt{sources.list} file if it asks for an install CD.
	\item Visit the web page on your server using your desktop's browser.
	\item Perform an \texttt{nmap} scan of your own machine and note the results.  You should find ports 80 and 22 open and all others closed.
	\item Configure your \texttt{iptables} firewall following the tutorial at \url{https://www.digitalocean.com/community/tutorials/how-to-set-up-a-firewall-using-iptables-on-ubuntu-14-04}
	\item Verify that you can access your VM using ssh and http.
	\item Perform another nmap scan.  What, if anything, has changed?
	\item Now modify your firewall to block http traffic and verify that it worked.
	\item Perform another nmap scan and note the results.
	
\end{enumerate}



\end{document}
