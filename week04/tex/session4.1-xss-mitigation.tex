% Beamer slide template prepared by Tom Clark <tom.clark@op.ac.nz>
% Otago Polytechnic
% Dec 2012

\documentclass[10pt]{beamer}
\usetheme{Dunedin}
\usepackage{graphicx}
\usepackage{fancyvrb}

\newcommand\codeHighlight[1]{\textcolor[rgb]{1,0,0}{\textbf{#1}}}

\title{Mitigating XSS Risk}

\author[IN618]{Security}
\institute[Otago Polytechnic]{
  Otago Polytechnic \\
  Dunedin, New Zealand \\
}
\date{}
\begin{document}

%----------- titlepage ----------------------------------------------%
\begin{frame}[plain]
  \titlepage
\end{frame}


\begin{frame}
	\frametitle{The problem}
	\begin{itemize}
		\item We saw last week that XSS vulnerabilities happen when:
			\begin{enumerate}
				\item We trust user input to be valid and safe;
				\item We output unsafe data to the browser without 
					processing it.
			\end{enumerate}
		\item We can also make the vulnerability worse by being too accepting
			of various types of HTTP requests.
	\end{itemize}

	It's important to recognise these patterns so that you know when to be 
	on the alert for XSS vulnerabilities.
\end{frame}

\begin{frame}[fragile]
	\frametitle{Vulnerable code}
	\begin{verbatim}
    <p> Hi, <?php echo($_COOKIE['user']); ?> </p> 
    <p><img src="picard.jpg" /></p>
    <p> <?php echo($_REQUEST['secure']); ?> </p>`
	\end{verbatim}
\end{frame}

\begin{frame}
	\frametitle{Step one: processing output}

	\begin{itemize}
		\item Much of the problem comes from outputting things
			like unwanted``\texttt{<script>}'' tags to the browser.
		\item Usually, we don't want to pass any HTML from user input
			back to the browser.
		\item It turns out that characters like ``\texttt{<}'' and
			``\texttt{>}'' have alternative representations.
		\item For eample, ``\texttt{>}'' can be represented as 
			``\texttt{\&gt;}''.
		\item These representations are safer to send to the users' browser.
	\end{itemize}
\end{frame}
\begin{frame}[fragile]
	\frametitle{In PHP:}
	Instead of 
	\begin{verbatim}
    <?php echo($unsafe-data); ?> 
	\end{verbatim}
	Use
	\begin{verbatim}
    <?php echo(htmlentities($unsafe-data)); ?> 
	\end{verbatim}
\end{frame}
\begin{frame}
	\frametitle{This doesn't solve everything}
	\begin{itemize}
		\item It's important to read the documentation\footnote{\url{http://php.net/htmlentities}} for \texttt{htmlentities()} to use it correctly.
		\item It may stil be possible to get some malicious code past
			\texttt{htmlentities()}.
		\item There are also situations where you may want to allow 
			characters that \texttt{htmlentities} transforms to
			pass through to the browser.
		\item The bottom line is that it \emph{improves} our vulnerability 
			exposure considerably.
	\end{itemize}
\end{frame}
\begin{frame}
	\frametitle{Step two: processing input}
	\begin{itemize}
		\item It's important to avoid sending malicious code to 
			browsers, but it's probably best to avoid accepting
			malicious code in input.
		\item One approach would be to write a function that filters 
			out everything that's ``bad''.
		\item This can work, but now we have to anticipate everything
			bad that could come in input.
		\item Another approach is to identify what we \emph{want} in 
			our input and to write functions that only accept that.
		\item This approach requires more work, but it's easier to get
			right.
	\end{itemize}
\end{frame}
\begin{frame}
	\frametitle{Example}
	\begin{itemize}
		\item In our sample code last week we had users enter
			their names.
		\item It's reasonable to place some conditions on user names.
			\begin{itemize}
				\item not more than 32 characters
				\item only letters, digits, - and \_
			\end{itemize}
		\item We can write a function to check that these
			conditions are met.
	\end{itemize}
\end{frame}

\begin{frame}[fragile]
	\frametitle{Username checker}

	\begin{verbatim}
    function legal_username($name) {
        $length_ok = strlen($name) < 33;
        $format_ok = preg_match("/^[a-zA-Z-_]+$/", $name);
        return $length_ok and $format_ok;
    }
	\end{verbatim}
\end{frame}
\begin{frame}[fragile]
	\frametitle{Using the username checker}

	\begin{verbatim}
    <?php 
     if(legal_username($_POST['un'])) {
      setcookie("user", $_POST['un'], mktime().time() + 7200);
     }
      ?>

	\end{verbatim}
\end{frame}

\end{document}

