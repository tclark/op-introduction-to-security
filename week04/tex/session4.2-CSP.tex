% Beamer slide template prepared by Tom Clark <tom.clark@op.ac.nz>
% Otago Polytechnic
% Dec 2012

\documentclass[10pt]{beamer}
\usetheme{Dunedin}
\usepackage{graphicx}
\usepackage{fancyvrb}

\newcommand\codeHighlight[1]{\textcolor[rgb]{1,0,0}{\textbf{#1}}}

\title{Content Security Policy}

\author[IN618]{Security}
\institute[Otago Polytechnic]{
  Otago Polytechnic \\
  Dunedin, New Zealand \\
}
\date{}
\begin{document}

%----------- titlepage ----------------------------------------------%
\begin{frame}[plain]
  \titlepage
\end{frame}


\begin{frame}
	\frametitle{The problem}
	
	Why might we be vulnerable to Cross Site Scripting?
	
\end{frame}

\begin{frame}
	\frametitle{The problem}
	
	Basically, we want to be able to supply resources like JavaScript 
	to our pages without allowing anyone else to do so.
	
	
\end{frame}

\begin{frame}
	\frametitle{Content Security Policy}

	\emph{Content Security Policy} (CSP) is a candidate standard that allows us to 
	specify the legitimate sources for scripts, styles, media and other resources that
	can be loaded into a web page. Resources from other locations are blocked.
	 
\end{frame}
\begin{frame}[fragile]
	\frametitle{Example}

	\begin{verbatim}
   Content-Security-Policy:    default-src 'self'; 
                               style-src 'self' http://cdn.example.com; 
                               script-src http://cdn.example.com;
	\end{verbatim}
	
	We specify the policy in our HTTP headers that we send with our web pages.
	
\end{frame}

\begin{frame}[fragile]
	\frametitle{How do we do this?}
	
	We can configure our web servers to send the headers automatically with every response, or
	we can specify the headers in our web application code.
	
	In PHP:
	\begin{verbatim}
	header("Content-Security-Policy: default-src 'self'; 
	        script-src http://cdn.example.com;")
	\end{verbatim}
	
	Note that we must specify headers before we send any other output to the client.
	
\end{frame}

\begin{frame}[fragile]
\frametitle{CSP directives}

\begin{verbatim}
default-src
script-src
object-src
style-src
img-src
media-src
frame-src
font-src
connect-src
\end{verbatim}




\end{frame}

\begin{frame}[fragile]
\frametitle{CSP Logging}

With CSP we can specify a url to which CSP violations will be logged. 

\begin{verbatim}
  Content-Security-Policy:    default-src 'self'; 
      report-uri: https://foo.org.nz/csp-report;
\end{verbatim}


\end{frame}

\begin{frame}[fragile]
	\frametitle{CSP Report-Onle}
	
	With CSP we can specify a url to which CSP violations will be logged. 
	
	\begin{verbatim}
	Content-Security-Policy-Report-Only:    default-src 'self'; 
	    report-uri: https://foo.org.nz/csp-report;
	\end{verbatim}
	
	
\end{frame}

\begin{frame}[fragile]
	\frametitle{Downsides}
	
	To use CSP to its fullest potential, you have to build your website
	so that it complies with the policies.
	
	\begin{itemize}
		\item No inline JavaScript
		\item No inline styles
		\item Any remote script sources (e.g. Google APIs) have to be identified
		\item etc.
	\end{itemize}
		
	This isn't a problem for new web sites, but legacy sites are likely to have problems.
\end{frame}

\begin{frame}[fragile]
	\frametitle{Downsides}
	
	
	Internet Explorer
	
	
\end{frame}

\begin{frame}[fragile]
	\frametitle{Downsides}
	
	
	Older Browsers
	
	
\end{frame}

\begin{frame}[fragile]
	\frametitle{Conclusion}
	
	
	CSP is a useful tool for reducing XSS risk, but it doesn't completely solve the problem, especially for older sites and older browsers.
	
	
\end{frame}

\begin{frame}[fragile]
	\frametitle{Demo}
	
	
    \url{http://xss.foo.org.nz/lab4.2}
    
    \vspace{10mm}
    Be sure to try this on a variety of browsers.
	
	
\end{frame}
\end{document}

