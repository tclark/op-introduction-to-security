% Beamer slide template prepared by Tom Clark <tom.clark@op.ac.nz>
% Otago Polytechnic
% Dec 2012

\documentclass[10pt]{beamer}
\usetheme{Dunedin}
\usepackage{graphicx}
\usepackage{fancyvrb}

\newcommand\codeHighlight[1]{\textcolor[rgb]{1,0,0}{\textbf{#1}}}

\title{Password Hashing}

\author[IN618]{Security}
\institute[Otago Polytechnic]{
  Otago Polytechnic \\
  Dunedin, New Zealand \\
}
\date{}
\begin{document}

%----------- titlepage ----------------------------------------------%
\begin{frame}[plain]
  \titlepage
\end{frame}


\begin{frame}
	\frametitle{The problem}
	\begin{itemize}
		\item It's clear that we need to encrypt users' passwords.
		\item We can use a one way hashing function to do this, since we don't ever need to decrypt it.
		\item Instead we hash the supplied password in the same way that we hashed the stored password
		\item If the two hashes match, then we authenticate the user.
	\end{itemize}

\end{frame}

\begin{frame}
	\frametitle{Naive hashing is not enough}
	\begin{itemize}
		\item If we have a database of hashed passwords, it's still 
			a bit too easy to break.
		\item This is especially true if our users have bad passwords.
	\end{itemize}
\end{frame}

\begin{frame}
	\frametitle{It's too easy to find passwords}
	\begin{itemize}
		\item Suppose we want to see if anyone is using the 
			password ``password''.
		\item We hash it and find that it hashes to \\
			\texttt{E201065D0554652615...5BC8EDCA469D72C2790E24152D0C1E2B6189}
		\item Now we scan the password database to look for a match.
	\end{itemize}
\end{frame}

\begin{frame}
	\frametitle{The solution:  add salt}
	\begin{itemize}
		\item We solve this problem by adding \emph{salt} to the hashed password.
		\item The process for generating a salted password is:
			\begin{enumerate}
				\item Start with the supplied password, e.g., \texttt{foo}.
				\item Generate some random bytes that we can represent
					as a string, e.g., \texttt{salt}.
				\item Prepend the salt string to the password and hash 
					the combined string.  Our hashed password is thus \\
					\texttt{ hashedpw = hash("saltfoo")}
				\item Now in our password store, we save the username,
					the salt, and the hashed password.
			\end{enumerate}
	\end{itemize}
\end{frame}
\begin{frame}
	\frametitle{Verifying the salted passwords}
	The verification process is:
	\begin{enumerate}
		\item Get a username/password combination to verify.
		\item Retrieve the salt and hashed password from your
			user data store.
		\item Hash the supplied password with the salt from the saved password.
		\item Check to see that the hashes match.
	\end{enumerate}
\end{frame}
\begin{frame}
	\frametitle{Conclusion}

	Do you see why this makes our password database harder to crack?
\end{frame}
					
\end{document}
