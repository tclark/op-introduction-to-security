\documentclass{article}
\usepackage{graphicx}
\usepackage{enumerate}
\usepackage{verbatim}
\usepackage{hyperref}
\usepackage[parfill]{parskip}
\usepackage[margin = 2.5cm]{geometry}

\usepackage[T1]{fontenc}


\begin{document}

\title{Exam Review Outline\\ IN618 Security}
\date{}
\maketitle

The final exam for 2016 will be held on Thursday, June 16. There will be one session at 1:00 PM and a second at 3:00 PM, both in D313. You should revise the following topics to prepare for the exam.

\textbf{Risk}

\begin{itemize}
	\item Know the definition of risk. (Basically, Probability of harmful event * degree of harm from event)
	\item Be able to calculate an ALE.
	\item Understand how an ALE informs budget calculations for risk reduction measures.
	\item Know the difference between quantitative and qualitative risk assessment.
	\item Be able to use qualitative risk assessments to prioritise security decisions.
\end{itemize}

\textbf{Password hashing}

\begin{itemize}
	\item Explain the algorithm for authenticating a user with hashed passwords.
	\item Understand the difference between hashing and two-way encryption methods. Why is hashing a good method for protecting passwords?
	\item Explain how using salt with hashed passwords improves their security.
\end{itemize}

\textbf{Common vulnerabilities: XSS, XSRF, SQL injection}

For each or these vulnerabilities:

\begin{itemize}
	\item Explain how the vulnerability can typically be exploited.
	\item Explain the nature of the mistakes in coding that lead to the vulnerability and how to fix them.
	\item Explain the risk associated with the vulnerability - what harm can result from its exploitation and how likelihood that it may be exploited. 
\end{itemize}


\textbf{Buffer overflows}

\begin{itemize}
	\item Explain why buffer overflow vulnerabilities are so serious.
	\item Explain how buffer overflow attacks work.
	\item Know what coding mistakes lead to buffer overflow vulnerabilities.
\end{itemize}

\textbf{Encryption}

\begin{itemize}
	\item Know the difference between symmetric key and asymmetric key encryption.
	\item Explain the challenge of public key distribution and verification.
	\item Know the difference between a self-signed key and one signed by a certififate authority(CA) key when setting up https. 
	\item Know the security advantages of using public key authentication with ssh.
\end{itemize}

\newpage

\textbf{Port scanning}

\begin{itemize}
	\item Explain how common port scanning methods work.
	\item Interpret sample port scan results (produced by nmap).
	\item List some uses for port scanning besides reconnoitring for an attack. 
\end{itemize}

\textbf{Firewalls}

\begin{itemize}
	\item Know what items of information can be used in packet filtering firewall rules.
	\item Interpret the meaning of example firewall rules (using \texttt{iptables})
	\item Explain the function of application firewalls and how they differ from packet filtering firewalls.
\end{itemize}

\end{document}
