\documentclass{article}
\usepackage{graphicx}

\usepackage[margin = 2.5cm]{geometry}




\begin{document}

\title{Lab 1.2: Risk Worksheet\\ IN618 Security}
\date{}
\maketitle

\section*{Instructions}
Work through the risk assessment exercises below and write your solutions \textbf{neatly}.  Be sure to show your work and/or justify your answers.

\subsection*{Exercise one}
Suppose that a qualitative analysis rates the following risks as below.

\vspace{5mm}

\begin{tabular}{l c c}
	Security Event & Event Probability & Resulting Harm \\
	Web Site Defacement & Medium & Low \\
	XSS Attack & High & Medium \\
	Buffer Overflow on Web Server & Low & High \\

\end{tabular}

Rate these risks from highest to lowest priority.

\vspace{25mm}

\subsection*{Exercise two}
A web site generates \$25,000 per hour in revenue. The probability of a web site outage in any given year is 10\% and such an outage would last 2 hours and cost \$1200 to correct. What is the Annual Loss Expectancy (ALE)?

\newpage

\subsection*{Exercise Three}
A small consulting firm works on one project at a time and stores project data on a single server. The value of this data is \$100,000 and a server failure would jeopardise up to 90\% of it. The probability of such a failure in any year is 0.20.  What is the ALE?

\vspace{75mm}

\subsection*{Exercise Four}
You have a \$3 million data centre located in an area at risk of flooding. A major flood that would destroy the data centre occurs once every hundred years. Compute the ALE.

\vspace{60mm}

Based on this ALE, would you recommend that the company spend \$35,000 per year to control this threat?







\end{document}
